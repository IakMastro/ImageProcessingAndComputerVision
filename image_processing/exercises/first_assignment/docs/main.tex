\documentclass{fphw}

\usepackage{ucs}

\usepackage[utf8x]{inputenc}
\usepackage[greek, english]{babel}
\usepackage{alphabeta}
\usepackage{lmodern}

\usepackage[T1]{fontenc}
\usepackage{graphicx}
\usepackage{booktabs}
\usepackage{enumerate}
\usepackage{float}

\usepackage{xcolor}
\usepackage{listings}

\definecolor{mygreen}{rgb}{0,0.6,0}
\definecolor{mygray}{rgb}{0.5,0.5,0.5}
\definecolor{mymauve}{rgb}{0.58,0,0.82}

\lstset{ %
  backgroundcolor=\color{white},   % choose the background color
  basicstyle=\footnotesize,        % size of fonts used for the code
  breaklines=true,                 % automatic line breaking only at whitespace
  captionpos=b,                    % sets the caption-position to bottom
  commentstyle=\color{mygreen},    % comment style
  escapeinside={\%*}{*)},          % if you want to add LaTeX within your code
  keywordstyle=\color{blue},       % keyword style
  stringstyle=\color{mymauve},     % string literal style
}

\usepackage[hidelinks]{hyperref}

\title{Πρώτη εργαστηριακή εργασία}
\author{Ιάκωβος Μαστρογιαννόπουλος}
\institute{Πανεπιστημιο Δυτικης Αττικης \\ Τμημα Μηχανικων Πληροφορικης και Υπολογιστων}
\class{Επεξεργασία Εικόνας}
\professor{Ευτύχιος Πρωτοπαπαδάκης}

\begin{document}
\includegraphics[width=25mm]{Figures/Logo}
\maketitle

\begin{abstract}
	Στην πρώτη εργαστηριακή εργασία, ζητείται να υλοποιηθούν δύο παραδείγματα επεξεργασίας εικόνας. Το πρώτο παράδειγμα, πρέπει να χρησιμοποιηθεί ο αλγόριθμος solarize
	για να δημιουργηθεί το φαινόμενο Sabbatier. Στο δεύτερο παράδειγμα, πρέπει να προστεθεί θόρυβος σε μία εικόνα, με την χρήση του salt and pepper και να τον αφαιρεί με χρήση φίλτρων.
	Υλοποιήθηκε σε γλώσσα προγραμματισμού Python, έκδοση 3.9.9 σε περιβάλλον Linux.
\end{abstract}

\newpage
\tableofcontents
\lstlistoflistings
\listoffigures
\listoftables

\newpage
\label{Chapter1}

\section{Χώροι χρώματος και κατάτμηση}

Σε αυτό το κεφάλαιο θα πραγματοποιηθεί ένα παράδειγμα ιστογράμματος και κατάτμησης για να μελετηθεί η διαφορά του RGB, του HSV και του συνδυασμού τους.

\subsection{Επιλογή εικόνας}

Η εικόνα η οποία επιλέχτηκε για να εκτελεστούν οι αλγόριθμοι πάνω της είναι η εικόνα του Σχήματος~\ref{fig:doctor}

\begin{figure}[H]
  \centering
  \includegraphics[width=100mm]{Figures/the_doctor}
  \caption[Τα 13 regenerations του Doctor]{Οι διάφορες μορφές (regenerations) του Doctor από την σειρά <<Doctor Who>> του BBC, 1963 - 1989 και 2005 - σήμερα}
  \label{fig:doctor}
\end{figure}

\subsubsection{3D plot της original RGB εικόνας}

Για να φορτωθεί η εικόνα στο script και να την θεωρήσει ως RGB χρειάζονται μερικές εντολές. Τα ακρώνυμα RGB σημαίνει Red (κόκκινο), Green (πράσινο) και Blue (Μπλε). Η OpenCV όμως τα διαβάζει ως BGR και θα πρέπει να γίνει η κατάλληλη μετατροπή σε RGB. Αυτό γίνετε με τον εξής τρόπο:

\begin{minted}{python}
image = cv2.imread('the_doctor.jpg')
image = cv2.cvtColor(image, cv2.COLOR_BGR2RGB)
\end{minted}

Για να πραγματοποιηθεί 3D plot, θα πρέπει να γίνει διαχωρισμός των καναλιών. Υπάρχει έτοιμη συνάρτηση στο OpenCV που το κάνει αυτόματα. Τα αποτελέσματα της συνάρτησης παρουσιάζονται στο Σχήμα~\ref{fig:rgb_3d_plot}.

\begin{figure}[H]
  \centering
  \includegraphics[width=100mm]{Figures/rgb_3d}
  \caption{RGB 3D Plot}
  \label{fig:rgb_3d_plot}
\end{figure}

\subsubsection{Ιστόγραμμα της RGB εικόνας}

Το ιστόγραμμα είναι ένα πολύ δυνατό εργαλείο στην επεξεργασία εικόνας. Απεικονίζει την διαφορά των τιμών μίας εικόνας ανά κανάλι. Μπορεί να υπολογισθεί πολύ εύκολα χρησιμοποιώντας την βιβλιοθήκη NumPy.

\begin{minted}{py}
for channel_id, channel in zip(channel_ids, channels):
  histogram, bin_edges = np.histogram(image[:, :, channel_id], bins=bins, range=(0, bins))
\end{minted}

Τροποποιώντας λίγο τον κωδικά, μπορεί να εισαχθεί σε ένα plot. Το αποτέλεσμα του plot παρουσιάζεται στο Σχήμα~\ref{fig:rgb_histogram}.

\begin{minted}{py}
def plot_histogram(image, labels, bins,
                   title, filename, channels=('red', 'green', 'blue'),
                   channel_ids=(0, 1, 2)) -> None:
  plt.xlim([0, 256])

  for label, channel_id, channel in zip(labels, channel_ids, channels):
    histogram, bin_edges = np.histogram(
      image[:, :, channel_id],
      bins=bins,
      range=(0, bins)
    )
    plt.plot(bin_edges[0:-1], histogram, label=label, color=channel)

  plt.xlabel("Color value")
  plt.ylabel("Pixels")
  plt.title(title)
  plt.legend(loc="best")
  plt.savefig(f"{filename}.png")
  plt.show()

plot_histogram(image, ('Red', 'Green', 'Blue'), 256, "RGB Histogram", "rgb_histogram")
\end{minted}

\begin{figure}[H]
  \centering
  \includegraphics[width=100mm]{Figures/rgb_histogram}
  \caption{RGB Histogram}
  \label{fig:rgb_histogram}
\end{figure}

\subsubsection{Κατάτμηση της RGB εικόνας}

Κατάτμηση είναι όταν μία εικόνα χωρίζεται σε μικρότερα τμήματα τα οποία είναι όμοια μαζί τους. Η κατάτμηση μπορεί να γίνει με πολλές μεθοδολογίες. Οι δύο που επιλέχτηκαν για το παράδειγμα είναι ο MeanShift και ο KMeans. Και οι δύο αλγόριθμοι είναι παρόμοιοι και κάνουν την ίδια δουλεία, εύρεση των clusters.

\begin{minted}{py}
def segment_image(image, color_code, channel_amount=3) -> None:
  shape = image.shape
  flat_image = np.reshape(image, [-1, channel_amount])

  # MeanShift segmentation
  bandwidth = estimate_bandwidth(flat_image, quantile=0.1, n_samples=100)
  meanshift = MeanShift(bandwidth=bandwidth, bin_seeding=True)
  meanshift.fit(flat_image)
  segmentation(flat_image, shape, color_code, channel_amount, meanshift, "MeanShift")

  # KMeans segmentation
  k_means = MiniBatchKMeans()
  k_means.fit(flat_image)
  segmentation(flat_image, shape, color_code, channel_amount, k_means, "KMeans")

segment_image(image, "RGB")
\end{minted}

\begin{figure}[H]
  \centering
  \includegraphics[width=100mm]{Figures/RGB_MeanShift}
  \caption{RGB MeanShift}
  \label{fig:rgb_meanshift}
\end{figure}

\begin{figure}[H]
  \centering
  \includegraphics[width=100mm]{Figures/RGB_KMeans}
  \caption{RGB KMeans}
  \label{fig:rgb_kmeans}
\end{figure}

Στα Σχήματα~\ref{fig:rgb_meanshift} και ~\ref{fig:rgb_kmeans} απεικονίζονται τα αποτελέσματα των αλγορίθμων.

\subsection{Μετατροπή σε HSV και παρουσίαση αποτελεσμάτων}

Το HSV είναι ένας εναλλακτικός τρόπος αναπαράστασης της εικόνας. Τα ακρώνυμα είναι Hue (απόχρωση), Saturation (κορεσμός) και Value (τιμή). Η μετατροπή από RGB σε HSV είναι η εξής:

\begin{minted}{py}
hsv_image = cv2.cvtColor(image, cv2.COLOR_RGB2HSV)
save_image("hsv", hsv_image)
\end{minted}

Η εικόνα σε HSV παρουσιάζεται στο Σχήμα~\ref{fig:hsv}. Να παρατηρηθεί πόσο πολύ έχει τροποποιηθεί η εικόνα. Στα Σχήματα~\ref{fig:hsv_meanshift} και \ref{fig:hsv_kmeans} παρουσιάζονται τα αποτελέσματα των αλγορίθμων MeanShift και KMeans στην HSV εικόνα. Παρατηρείτε ότι τα αποτελέσματα έχουν επίσης τροποποιηθεί.

\begin{figure}[H]
  \centering
  \includegraphics[width=100mm]{Figures/hsv}
  \caption{Η HSV εικόνα}
  \label{fig:hsv}
\end{figure}

\begin{figure}[H]
  \centering
  \includegraphics[width=100mm]{Figures/HSV_MeanShift}
  \caption{HSV MeanShift}
  \label{fig:hsv_meanshift}
\end{figure}

\begin{figure}[H]
  \centering
  \includegraphics[width=100mm]{Figures/HSV_KMeans}
  \caption{HSV KMeans}
  \label{fig:hsv_kmeans}
\end{figure}

\subsection{Σχολιασμός αποτελεσμάτων RGB και HSV}
\label{paratiriseis}

Τα αποτελέσματα μεταξύ των δύο αλγορίθμων είτε ο KMeans είτε ο MeanShift είναι πολύ διαφορετικά μεταξύ τους με RGB και με το HSV. Ο MeanShift παράγει πολύ καλά με το RGB μοντέλο, ένω ο KMeans με το HSV μοντέλο.

\subsubsection{Συνδιασμός RGB και HSV}

Μπορεί να εξελιχθεί λίγο παραπάνω το ερώτημα συνδυάζοντας τα δύο μοντέλα και περνώντας τα από τους δύο αλγορίθμους. Ο αλγόριθμος που συνδυάζει τα δύο μοντέλα είναι ο εξής:

\begin{minted}{py}
combined_image = np.concatenate((image, hsv_image), axis=2)
\end{minted}

Στα Σχήματα~\ref{fig:rgb_hsv_meanshift} και ~\ref{fig:rgb_hsv_kmeans} φαίνονται τα αποτελέσματα των αλγορίθμων. Τα αποτελέσματα φαίνονται πολύ πιο καλύτερα από όταν ήταν ξεχωριστά τα μοντέλα.

\begin{figure}[H]
  \centering
  \includegraphics[width=100mm]{Figures/RGB&HSV_MeanShift}
  \caption{RGB και HSV MeanShift}
  \label{fig:rgb_hsv_meanshift}
\end{figure}

\begin{figure}[H]
  \centering
  \includegraphics[width=100mm]{Figures/RGB&HSV_KMeans}
  \caption{RGB και HSV KMeans}
  \label{fig:rgb_hsv_kmeans}
\end{figure}

\subsection{Silhouette Coefficient}

Ο Silhouette Coefficient ή silhouette score είναι ένας αριθμός που καθορίζει το πόσο καλό είναι μία clustering τεχνική. Το εύρος τιμών του είναι από το $ -1 $ εώς το $ 1 $, όπου:

\begin{itemize}
  \item 1: τα clusters μεταξύ τους είναι αρκετά μακρυά και εύκολα αναγνωρίσιμα
  \item 0: τα clusters είναι τελείως διαφορετικά μεταξύ τους
  \item -1: τα clusters έχουν ρυθμιστεί λάθος
\end{itemize}

Μαθηματικά:

\begin{equation}
  SilhouetteScore = \frac{(b - a)}{max(a, b)}
\end{equation}

όπου:

\begin{itemize}
  \item το a είναι η μέση απόσταση των inter-cluster από κάθε σημείο
  \item το b είναι η μέση απόσταση όλων των clusters
\end{itemize}

\begin{table}[H]
  \centering
	\begin{tabular}{ | p{4cm} | p{4cm} | p{6cm} | }
		\hline
		\textbf{Όνομα εικόνας} & \textbf{Αριθμός clusters} & \textbf{Αποτέλεσμα Silhouette} \\
		\hline
		RGB MeanShift & 5 & 0.43829281680682963 \\
		\hline
		RGB KMeans & 8 & 0.39261542176396413 \\
		\hline
		HSV MeanShift & 6 & 0.27099304439301336 \\
		\hline
    HSV KMeans & 8 & 0.3545841420133397 \\
    \hline
    RGB και HSV MeanShift & 4 & 0.262483435384828 \\
    \hline
    RGB και HSV KMeans & 8 & 0.33426919705784475 \\
    \hline
	\end{tabular}
  \caption[Αποτελέσματα Silhouette]{Tα αποτελέσματα που γύρισε ο αλγόριθμος φαίνονται στον παρακάτω πίνακα. Οι παρατηρήσεις που έγιναν στο Κεφάλαιο~\ref{paratiriseis} φαίνονται να επιβεβαιώνονται με τα αποτελέσματα.}
  \label{tab:silhouette}
\end{table}

\label{Chapter2}

\section{Μορφολογία και ανίχνευση ακμών}

Στο αυτό το κομμάτι της εργασίας, το ζητούμενο είναι να γίνει ανίχνευση των ακμών με την χρήση της τεχνικής Canny σε μία grayscale εικόνα. Στην συνέχεια, πρέπει να προστεθεί και να παρατηρηθούν οι αλλαγές.

\subsection{Τεχνική Canny}

H τεχνική ανίχνευση ακμών Canny είναι ένας διάσημος αλγόριθμος ανίχνευσης ακμών. Για να ολοκληρωθεί και να βρει τις ακμές περνάει από μερικά στάδια. Στην Python, παράγεται έτοιμο. Ο κώδικας είναι ο εξής:

\begin{minted}{py}
def apply_edges(image, t_lower=100, t_upper=200, aperture_size=5, L2Gradient=True) -> any:
  return cv2.Canny(
    image, t_lower, t_upper,
    apertureSize=aperture_size, L2gradient=L2Gradient
  )
\end{minted}

Η φωτογραφία που επιλέχτηκε για να γίνει η ανίχνευση των ακμών βρίσκετε στο Σχήμα~\ref{fig:guitar_god}, ενώ τα αποτελέσματα του αλγορίθμου στο Σχήμα~\ref{fig:original}.

\begin{figure}[H]
  \centering
  \includegraphics[width=100mm]{Figures/david_gilmour}
  \caption[David Gilmour]{Ο David Gilmour, γνωστός για την συμμετοχή του στο συγκρότημα Pink Floyd και ως τρομερός μουσικός}
  \label{fig:guitar_god}
\end{figure}

\begin{figure}[H]
  \centering
  \includegraphics[width=100mm]{Figures/Original}
  \caption[Η σύγκριση του Original με την τεχνική canning]{Η σύγκριση του Original με την τεχνική canning. Να παρατηρηθεί ότι ο ίδιος φαίνεται ελάχιστα και πιάνει πολύ background noise ο αλγόριθμος}
  \label{fig:original}
\end{figure}

\subsection{Γκαουσιανός θόρυβος}

Εφαρμογή αλγορίθμου για προσθήκη Γκαουσιανού θορύβου στην εικόνα.

\begin{minted}{py}
def add_noise_and_remove_it(image) -> any:
  gaussian_noise = np.random.normal(0, 1, image.size)
  gaussian_noise = gaussian_noise.reshape(image.shape[0], image.shape[1]).astype('uint8')
  noise = cv2.add(image, gaussian_noise)
  return cv2.medianBlur(noise, 5)
\end{minted}

Στο Σχήμα~\ref{fig:restored}, μπορεί να βγει το συμπέρασμα ότι προσθέτοντας θόρυβο στην εικόνα, μπορεί να βοηθήσει την τεχνική Canny για να εντοπίσει πιο σωστά τις ακμές της εικόνας.

\begin{figure}[H]
  \centering
  \includegraphics[width=100mm]{Figures/Restored}
  \caption[Η σύγκριση της επαναφοράς της εικόνας με την τεχνική canning]{Η σύγκριση της επαναφοράς της εικόνας με την τεχνική canning. Να παρατηρηθεί ότι ο ίδιος φαίνεται πολύ καλύτερα και μπορεί να παρατηρηθεί η μορφή του γνωστού κιθαρίστα}
  \label{fig:restored}
\end{figure}

\subsection{Συνδυασμός μορφολογίας με σύγκριση ακμών}

Μορφολογία στην επεξεργασία εικόνας είναι το πως αναπαριστάτε η κάθε φιγούρα μέσα στην εικόνα. Υπάρχουν δύο μορφολογικοί τελεστές που χρησιμοποιήθηκαν σε αυτή την εργασία είναι ο erosion και ο dilation.

\subsubsection{Μορφολογικός Τελεστής Erosion}

\begin{equation}
  Α \ominus Β = \{ z | (Β)_z \subseteq Α \}
\end{equation}

\begin{figure}[H]
  \centering
  \includegraphics[width=100mm]{Figures/Erosion}
  \caption[Η σύγκριση της επαναφοράς της εικόνας αφού έχει περαστεί απο erosion με την τεχνική canning]{Η σύγκριση της επαναφοράς της εικόνας αφού έχει περαστεί από erosion με την τεχνική canning. Η μορφή του κιθαρίστα είναι ακόμα πιο καθαρή από πριν, βεβαία περνάει και λίγο από το background noise μέσα στις ακμές του}
  \label{fig:erosion}
\end{figure}

\newpage
\subsubsection{Μορφολογικός Τελεστής Dilation}

\begin{equation}
  Α \oplus Β = \{ z | (\hat{Β})_z \cap Α \}
\end{equation}

\begin{figure}[H]
  \centering
  \includegraphics[width=100mm]{Figures/Dilation}
  \caption[Η σύγκριση της επαναφοράς της εικόνας αφού έχει περαστεί απο dilation με την τεχνική canning]{Η σύγκριση της επαναφοράς της εικόνας αφού έχει περαστεί από dilation με την τεχνική canning. Η μορφή του κιθαρίστα είναι αρκετά ίδια με το erosion, με την διαφορά ότι έχει παραπάνω θόρυβο}
  \label{fig:dilation}
\end{figure}

\subsection{Σύγκριμα αποτελεσμάτων με την χρήση αλγορίθμων}

Για το πόσο καλά αποτελέσματα παράγει ο αλγόριθμος, θα ακολουθούν διάφορες τεχνικές. Για να γίνει οποιαδήποτε σύγκριση όμως χρειάζεται να γίνει μετατροπή των εικόνων σε ιστογράμματα για να είναι πιο επεξεργάσιμοι από τους καθημερινούς μοντέρνους υπολογιστές. Ο αλγόριθμος δοκιμάστηκε σε σταθερό υπολογιστή με Ryzen 5 2600 και 8 GB ram και είχε ελάχιστη καθυστέρηση.

\subsubsection{Ευκλειδική Διαφορά}

Η πρώτη διαφορά που θα μελετηθεί είναι η Ευκλειδική. Ο αλγόριθμος είναι ο εξής:

\begin{minted}{py}
euclidean = 0
i = 0
while i < len(histogram_original) and i < len(histogram_edges):
  euclidean += (histogram_original[i] - histogram_edges[i]) ** 2
  i += 1
\end{minted}

\begin{table}[H]
  \centering
	\begin{tabular}{ | p{8cm} | p{8cm} | }
		\hline
		\textbf{Όνομα εικόνας} & \textbf{Αποτέλεσμα Ευκλειδικής Διαφοράς} \\
    \hline
    Original εικόνα & 1.3595214842916974 \\
    \hline
    Restored εικόνα & 1.4140797849329154 \\
    \hline
    Restored εικόνα με Erosion & 1.4136366487406054 \\
    \hline
    Restored εικόνα με Delation & 1.4135856679087746 \\
    \hline
	\end{tabular}
  \caption{Αποτελέσματα Ευκλειδικής διαφοράς}
  \label{tab:euclidean}
\end{table}

\newpage
\subsubsection{$ X^2 $ Διαφορά}

Η δεύτερη διαφορά είναι η $ X ^ 2 $ διαφορά. Δίνετε έτοιμο από το cv2.

\begin{table}[H]
  \centering
	\begin{tabular}{ | p{8cm} | p{8cm} | }
		\hline
		\textbf{Όνομα εικόνας} & \textbf{Αποτέλεσμα $ X^2 $ Διαφοράς} \\
    \hline
    Original εικόνα & 55.54426650712792 \\
    \hline
    Restored εικόνα & 9917.798371611698 \\
    \hline
    Restored εικόνα με Erosion & 1213.5299873964364 \\
    \hline
    Restored εικόνα με Delation & 7.813439429146478 \\
    \hline
	\end{tabular}
  \caption{Αποτελέσματα $ X^2 $ διαφοράς}
  \label{tab:chi_squared}
\end{table}

\subsubsection{Bhattacharyya Διαφορά}

Η τρίτη διαφορά είναι η Bhattacharyya διαφορά. Δίνετε έτοιμο από το cv2.

\begin{table}[H]
  \centering
	\begin{tabular}{ | p{8cm} | p{8cm} | }
		\hline
		\textbf{Όνομα εικόνας} & \textbf{Αποτέλεσμα Bhattacharyya Διαφοράς} \\
    \hline
    Original εικόνα & 0.9437122663789763 \\
    \hline
    Restored εικόνα & 0.9962561700845681 \\
    \hline
    Restored εικόνα με Erosion & 0.993916915962169 \\
    \hline
    Restored εικόνα με Delation & 0.9944792139506231 \\
    \hline
	\end{tabular}
  \caption{Αποτελέσματα Bhattacharyya διαφοράς}
  \label{tab:bhattacharyya}
\end{table}

\subsubsection{Correlation Διαφορά}

Η τέταρτη διαφορά είναι η correlation διαφορά. Δίνετε έτοιμο από το cv2.

\begin{table}[H]
  \centering
	\begin{tabular}{ | p{8cm} | p{8cm} | }
		\hline
		\textbf{Όνομα εικόνας} & \textbf{Αποτέλεσμα Correlation Διαφοράς} \\
    \hline
    Original εικόνα & 0.05002108348803092 \\
    \hline
    Restored εικόνα & -0.028392138047218348 \\
    \hline
    Restored εικόνα με Erosion & -0.022036753262860056 \\
    \hline
    Restored εικόνα με Delation & -0.030544818302308076 \\
    \hline
	\end{tabular}
  \caption{Αποτελέσματα Correlation διαφοράς}
  \label{tab:correlation}
\end{table}

\subsubsection{Intersection Διαφορά}

Η τελευταία διαφορά είναι η Intersection διαφορά. Δίνετε έτοιμο από το cv2.

\begin{table}[H]
  \centering
	\begin{tabular}{ | p{8cm} | p{8cm} | }
		\hline
		\textbf{Όνομα εικόνας} & \textbf{Αποτέλεσμα Intersection Διαφοράς} \\
    \hline
    Original εικόνα & 0.08144057751633227 \\
    \hline
    Restored εικόνα & 0.0007671799830859527 \\
    \hline
    Restored εικόνα με Erosion & 0.0008191215456463397 \\
    \hline
    Restored εικόνα με Delation & 0.008301599882543087 \\
    \hline
	\end{tabular}
  \caption{Αποτελέσματα Intersection διαφοράς}
  \label{tab:intersection}
\end{table}

\label{Chapter3}

\section*{Βιβλιογραφία}

\begin{itemize}
	\item Ψηφιακή Επεξεργασία Εικόνας (R. C. Gonzalez, R. E. Woods, Απόδοση Σ. Κόλλιας, εκδόσεις Τζιόλα, 2021)
	\item Πιθανότητες και Στατιστική (Ν. Μυλωνάς, Β. Παπαδόπουλος, εκδόσεις Τζιόλα, 2018)
	\item Σήματα και Συστήματα (Π. Φωτόπουλος, Α. Βελώνη, εκδόσεις Σύχρονη Εκδότικη, 2008)
\end{itemize}

\end{document}
