\label{Chapter2}

\section{Μέρος δεύτερο: διαχείριση θορύβου και επιλογή φίλτρων}

Σε αυτό το μέρος της εργασίας, πρόκειται να γίνει υποβάθμιση της εικόνας προσθέτοντας θόρυβο στην αρχική εικόνα και στην συνέχεια να γίνει η επιλογή των καταλλήλων φίλτρων για να επανέλθει στην αρχική της κατάσταση.
Θόρυβος στην επιστήμη των σημάτων ονομάζονται τα στοχαστικά σήματα τα οποία έχουν τυχαίες τιμές.
Στην επεξεργασία εικόνας, το σήμα αντιστοιχείται με το pixel στην οθόνη και ο θόρυβος με τις φυσικές καταστάσεις που γίνετε τριγύρω από την φωτογραφία.
Για παράδειγμα, έστω ότι ένας φωτογράφος τραβάει φωτογραφία ένα ποτάμι. Το ποτάμι δεν θα είναι πότε σταθερό.
Θα πρέπει να μπορεί να βρίσκεται σε θέση να αφαιρέσει τον θόρυβο στην φωτογραφία του με την χρήση των καταλλήλων φίλτρων.
Για την εργασία, θα πρέπει να βρεθεί ένας τρόπος ώστε να υποβαθμιστεί η original εικόνα, με σκοπό να επαναφερθεί στην αρχική της κατάσταση.

\subsection{Υποβάθμιση εικόνας με την χρήση θορύβου}

Για να γίνει η υποβάθμιση της εικόνας, θα πρέπει να προστεθεί ο θόρυβος.

\begin{problem}
  Ποιοι είναι γνωστοί και εύκολοι αλγόριθμοι που μπορούν να υλοποιηθούν για να προστεθεί ο θόρυβος;
\end{problem}

Οι δύο εύκολοι αλγόριθμοι που μπορούν να χρησιμοποιηθούν είναι ο \textbf{salt and pepper} και ο \textbf{poisson}

\subsubsection{Υλοποίηση του αλγορίθμου salt and pepper}

Ο αλγόριθμος salt and pepper είναι ένας αλγόριθμος ο οποίος δουλεύει πολύ καλά σε ασπρόμαυρες φωτογραφίες.
Όπως φαίνεται και από το όνομα, το salt (δηλαδή το αλάτι, το άσπρο) βρίσκει όλα τα άσπρα σημεία του pepper (πιπέρι, δηλαδή μαύρο) και το pepper το αντίθετο. \par
Ο αλγόριθμος μπορεί να υλοποιηθεί με την χρήση του NumPy και του random. Αυτό που κάνει ο αλγόριθμος με λίγα λόγια είναι να φτιάχνει ένα αντίγραφο της εικόνας.
Στην συνέχεια, παίρνει έναν τυχαίο αριθμό, τον οποίο τον ελέγχει με ένα όρισμα. Μαθηματικά:

\begin{equation}
  new\_image[i][j] = \Bigg\{
    \begin{tabular}{ccc}
      0 εάν $ random\_number < probability $ \\
      255 εάν $ random\_number > (1 - probability) $ \\
      αλλιώς image[i][j]
    \end{tabular}
\end{equation}

Όπου:

\begin{itemize}
  \item $ noise\_image $ είναι το αντίγραφο της εικόνας
  \item $ probability $ είναι το όρισμα που δέχεται
  \item $ image $ είναι η εικόνα η ίδια
\end{itemize}

Η υλοποίηση στην Python:

\begin{lstlisting}[language=Python, caption=Salt And Pepper]
def salt_and_pepper(image, probability):
  noise_image = np.zeros(image.shape, np.uint8)

  for row in range(image.shape[0]):
    for col in range(image.shape[1]):
      rand = random()
      if rand < probability:
        noise_image[row][col] = 0
      elif rand > (1 - probability):
        noise_image[row][col] = 255
      else:
        noise_image[row][col] = image[row][col]

  return noise_image
\end{lstlisting}

\subsubsection{Υλοποίηση του αλγορίθμου poisson}

\begin{problem}
  ``Αν ένα φαινόμενο επαναλαμβάνεται τυχαία στον χρόνο (ή χώρο) με μέσο αριθμό επαναλήψεων λ σε ένα χρονικό (χωρικό) διάστημα, η συνάρτηση πιθανότητας της τυχαίας μεταβλητής $ X $,
    η οποία μπορεί να είναι ο αριθμός επαναλήψεων στο διάστημα αυτό.

  \begin{equation}
    f_x(x) = e^-l * \frac{\lambda^x}{x!}, x = 0, 1, ...
  \end{equation}
  Στην περίπτωση αυτή, λέμε ότι η $ X $ ακολουθεί \textbf{κατανομή Poisson} με παράμετρο $ \lambda $ και γράφουμε''

  \begin{equation}
    X ~ P_o(\lambda)
  \end{equation}

  (Ν. Μυλωνάς, Β. Παπαδόπουλος)
\end{problem}

Στην επεξεργασία εικόνας, μπορεί να προστεθεί θόρυβος από την κατανομή του Poisson. To NumPy το έχει ήδη εφαρμοσμένο.

\begin{lstlisting}[language=Python, caption=Poisson]
def poisson(image):
  noise_image = np.random.poisson(image).astype(np.uint8)
  return image + noise_image
\end{lstlisting}

\subsection{Επαναφορά της εικόνας μέσω φίλτρων}

\begin{problem}
  ``Ο όρος φίλτρο έχει προέλθει από την επεξεργασία στο πεδίο της συχνότητας όπου αναφέρεται στην αποδοχή ή την απόρριψη συγκεκριμένων συχνοτήτων μιας εικόνας.''

  (R. C. Gonzalez, R. E. Woods)
\end{problem}

Για το συγκεκριμένο παράδειγμα, θα πρέπει να αφαιρεθούν με την χρήση τριών φίλτρων. Τα τρία φίλτρα τα οποία χρησιμοποιούνται είναι το \textbf{Γκαουσιανό φίλτρο}, το \textbf{φίλτρο του μέσου όρου} και το \textbf{φίλτρο της ενδιάμεσης τιμής}.

\subsubsection{Γκαουσιανό φίλτρο}

\begin{problem}
  ``Αν το πλήθος των κλάσεων είναι πολύ μεγάλο (τείνει στο άπειρο) και το πλάτος κλάσεων είναι πολύ μικρό (τείνει στο μηδέν), τότε τα πολύγωνα συχνοτήτων παίρνουν τη μορφή μίας ομαλής καμπύλης, οι οποίες ονομάζονται \textbf{καμπύλες συχνοτήτων} ή \textbf{καμπύλες σχετικών συχνοτήτων}.
  Σε πολλές περιπτώσεις μεταβλητών η καμπύλη σχετικών συχνοτήτων έχει μία κωδωνοειδή μορφή. Η καμπύλη αυτή ονομάζεται \textbf{κανονική κατανομή}.''

  (Ν. Μυλωνάς, Β. Παπαδόπουλος)
\end{problem}

Το Γκαουσιανό φίλτρο είναι ένα φίλτρο το οποίο χρησιμοποιεί την κανονική κατανομή για να επαναφέρει την εικόνα στην αρχική της κατάσταση. Η βιβλιοθήκη cv2 έχει μερικές συναρτήσεις που υλοποιούν το Γκαουσιανό φίλτρο.

\begin{lstlisting}[language=Python, caption=Gauss Sharp]
def calc_gauss_sharp(image):
  kernel = np.ones((5, 5), np.float32) / 25
  gauss_image = cv2.GaussianBlur(image, (5, 5), 0)

  kernel = np.array(
    [
      [0, -1, 0],
      [-1, 5, -1],
      [0, -1, 0]
    ]
  )

  return cv2.filter2D(gauss_image, -1, kernel)
\end{lstlisting}

\subsubsection{Φίλτρο του μέσου όρου}

Υπάρχουν πολλά φίλτρα μέσου όρου, αλλά αυτό το οποίος θα χρησιμοποιηθεί είναι το φίλτρο αριθμητικού μέσου.

\begin{problem}
  ``Το φίλτρο αριθμητικού μέσου αποτελεί και το πιο απλό από τα φίλτρα υπολογισμού του μέσου όρου τιμής.''
  (R. C. Gonzalez, R. E. Woods) \par
  Έστω $ S_{xy} $ το σύνολο των συντεταγμένων μίας εικόνας $ m * n $ διαστάσεων που έχει κέντρο το $ (x, y) $ Τότε:.

  \begin{equation}
    \hat{f}(x, y) = \frac{1}{m * n} * \sum_{(r, c) \in S_{xy}} g(r, c)
  \end{equation}
\end{problem}

\begin{lstlisting}[language=Python, caption=Average Sharp]
def calc_average_sharp(image):
  kernel = np.array(
    [
      [0, -1, 0],
      [-1, 5, -1],
      [0, -1, 0]
    ]
  )
  return cv2.filter2D(image, -1, kernel)
\end{lstlisting}

\subsubsection{Φίλτρο της ενδιάμεσης τιμής}

\begin{problem}
  ``Το πιο γνωστό φίλτρο στατιστικής διάταξης είναι το φίλτρο ενδιάμεσης τιμής που αντικαθιστά την τιμή ενός εικονοστοιχείου με την τιμή διάμεσου των επιπέδων έντασης που ανήκουν στη γειτονιά αυτού του εικονοστοιχείου.''
  (R. C. Gonzalez, R. E. Woods)

  \begin{equation}
    \hat{f}(x, y) = median\{g(r, c)\}
  \end{equation}
\end{problem}

\begin{lstlisting}[language=Python, caption=Median Blur]
def calc_median(image)
  median_blur = cv2.medianBlur(image, 5)
\end{lstlisting}

\subsection{Εφαρμογή αλγορίθμων}

\subsubsection{Εφαρμογή θορύβων}

Οι αλγόριθμοι πρόκειται να εφαρμοστούν πάνω στο Σχήμα~\ref{fig:collins}.
Στον πίνακα~\ref{tab:noise} απεικονίζονται οι νέες εικόνες με τον θόρυβο.
Παρατηρείται ότι και στις δύο περιπτώσεις, η αρχική εικόνα αλλάζει πάρα πολύ, κυρίως στον Poisson.

\begin{figure}[H]
  \centering
  \includegraphics[width=100mm]{Figures/face_value}
  \caption[Phil Collins - Face Value]{Δύο φωτογραφίες του διάσημου τραγουδιστή και drummer των Genesis, Phil Collins που τραβήχτηκαν σε δύο χρονικές περιόδους για το πρώτο του solo album, Face Value, 1981 και 2015}
  \label{fig:collins}
\end{figure}

\begin{table}[H]
  \centering
  \begin{tabular}{| p{8cm} | p{8cm}|}
    \hline
    \textbf{Salt and Paper} & \textbf{Poisson} \\
    \hline
    \includegraphics[width=\linewidth]{Figures/sp_noise} &
    \includegraphics[width=\linewidth]{Figures/poisson_noise} \\
    \hline
  \end{tabular}
  \caption{Εφαρμογή των αλγορίθμων του θορύβου στις εικόνες}
  \label{tab:noise}
\end{table}

\subsubsection{Εφαρμογή φίλτρων}

Στον πίνακα~\ref{tab:sp} φαίνονται τα αποτελέσματα από τα φίλτρα στην εικόνα με τον θόρυβο Salt and Pepper, ένω στον πίνακα~\ref{tab:poisson} τα αποτελέσματα από τον Poisson.

\begin{table}[H]
  \centering
  \begin{tabular}{| p{8cm} | p{8cm}|}
    \hline
    \textbf{Φίλτρο} & \textbf{Εικόνα} \\
    \hline
    Γκαουσιανό φίλτρο &
    \includegraphics[width=\linewidth]{Figures/sp_gauss_sharp} \\
    \hline
    Φίλτρο μέσου όρου &
    \includegraphics[width=\linewidth]{Figures/sp_average_sharp} \\
    \hline
    Φίλτρο διάμεσης τιμής &
    \includegraphics[width=\linewidth]{Figures/sp_median_blur} \\
    \hline
  \end{tabular}
  \caption{Εφαρμογή φίλτρων στην είκονα με τον θόρυβο Salt and Paper}
  \label{tab:sp}
\end{table}

\begin{table}[H]
  \centering
  \begin{tabular}{| p{8cm} | p{8cm}|}
    \hline
    \textbf{Φίλτρο} & \textbf{Εικόνα} \\
    \hline
    Γκαουσιανό φίλτρο &
    \includegraphics[width=\linewidth]{Figures/poisson_gauss_sharp} \\
    \hline
    Φίλτρο μέσου όρου &
    \includegraphics[width=\linewidth]{Figures/poisson_average_sharp} \\
    \hline
    Φίλτρο διάμεσης τιμής &
    \includegraphics[width=\linewidth]{Figures/poisson_median_blur} \\
    \hline
  \end{tabular}
  \caption{Εφαρμογή φίλτρων στην εικόνα με τον θόρυβο Poisson}
  \label{tab:poisson}
\end{table}

\subsection{Σύγκριση με την χρήση αλγορίθμων}

Για το τελευταίο βήμα, θα χρειαστεί να γίνει σύγκριση των εικόνων μεταξύ (α) της πρωτότυπης εικόνας με την εξομάλυνση και (β) της εικόνας με τον θόρυβο με την εξομάλυνση.
Οι δύο αλγόριθμοι που θα χρησιμοποιηθούν είναι ο αλγόριθμος \textbf{Structural Similarity Index (SSIM)} και ο \textbf{Mean Square Error}.

\subsubsection{Υλοποίηση του αλγορίθμου SSIM}

Ο αλγόριθμος \textbf{Structural Similarity Index (SSIM)} βρίσκει την διαφορά μεταξύ δύο παρόμοιων εικόνων. Στο skimage.metrics ύπαρχει έτοιμη σύναρτηση όνοματι ssim που το υλοποίει.

\begin{lstlisting}[language=Python, caption=SSIM Algorithm]
def calc_simil(name, original_image, image):
  simil_score, _ = ssim(original_image, image, full=True)
  print(f"{name} SSIM score is: {simil_score}")
\end{lstlisting}

\subsubsection{Υλοποίηση του αλγορίθμου Mean Square Error}

Ο αλγόριθμος \textbf{Mean Square Error} είναι ένας μετρητής που βρίσκει τον μέσο όρο των λαθών της δύναμης. Ο αλγόριθμος μπορεί να υλοποιηθεί πολύ απλά με το NumPy.

\begin{lstlisting}[language=Python, caption=Mean Squared Error Algorithm]
def calc_mean(name, original_image, image):
  mean_squared_error = np.square(np.subtract(original_image, image)).mean()
  print(f"{name} Mean Squared Error is: {mean_squared_error}")
\end{lstlisting}

\newpage
\subsubsection{Αποτελέσματα}

Τα αποτελέσματα μπορούν να φανούν στους πίνακες~\ref{tab:results_sp_no_noise},~\ref{tab:results_sp_noise},~\ref{tab:results_poisson_no_noise} και~\ref{tab:results_poisson_noise}.

\begin{table}[H]
  \centering
  \begin{tabular}{| p{2cm} | p{7cm} | p{6.5cm} |}
  \hline
  \textbf{Φίλτρο} & \textbf{Αποτέλεσμα SSIM} & \textbf{Αποτέλεσμα Mean Square Error} \\
  \hline
  Γκαουσιανό φίλτρο & 0.22224864314683754 & 95.79599344863892 \\
  \hline
  Φίλτρο μέσης τιμής & 0.1008267520614758 & 94.49013601462973 \\
  \hline
  Φίλτρο διάμεσης τιμής & 0.7457517695290394 & 40.69571883331589 \\
  \hline
  \end{tabular}
  \caption{Τα αποτελέσματα των συγκρίσεων μετάξυ του πρωτότυπου και της εξομάλυνσης του Salt and Pepper}
  \label{tab:results_sp_no_noise}
\end{table}

\begin{table}[H]
  \centering
  \begin{tabular}{| p{2cm} | p{7cm} | p{6.5cm} |}
  \hline
  \textbf{Φίλτρο} & \textbf{Αποτέλεσμα SSIM} & \textbf{Αποτέλεσμα Mean Square Error} \\
  \hline
  Γκαουσιανό φίλτρο & 0.5256954158313445 & 94.97967477171832 \\
  \hline
  Φίλτρο μέσης τιμής & 0.8065523248733352 & 73.49161068659988 \\
  \hline
  Φίλτρο διάμεσης τιμής & 0.0813248901272799 & 53.02096866552962 \\
  \hline
  \end{tabular}
  \caption{Τα αποτελέσματα των συγκρίσεων μετάξυ της εικόνας με τον θόρυβο και της εξομάλυνσης του Salt and Pepper}
  \label{tab:results_sp_noise}
\end{table}

\begin{table}[H]
  \centering
  \begin{tabular}{| p{2cm} | p{7cm} | p{6.5cm} |}
  \hline
  \textbf{Φίλτρο} & \textbf{Αποτέλεσμα SSIM} & \textbf{Αποτέλεσμα Mean Square Error} \\
  \hline
  Γκαουσιανό φίλτρο & 0.43015755824644086 & 105.33047737502233 \\
  \hline
  Φίλτρο μέσης τιμής & 0.12037766178876502 & 97.6301775147929 \\
  \hline
  Φίλτρο διάμεσης τιμής & 0.5137574899674178 & 108.9787480989354 \\
  \hline
  \end{tabular}
  \caption{Τα αποτελέσματα των συγκρίσεων μετάξυ του πρωτότυπου και της εξομάλυνσης του Poisson}
  \label{tab:results_poisson_no_noise}
\end{table}

\begin{table}[H]
  \centering
  \begin{tabular}{| p{2cm} | p{7cm} | p{6.5cm} |}
  \hline
  \textbf{Φίλτρο} & \textbf{Αποτέλεσμα SSIM} & \textbf{Αποτέλεσμα Mean Square Error} \\
  \hline
  Γκαουσιανό φίλτρο & 0.7198557213172727 & 75.2685503882174 \\
  \hline
  Φίλτρο μέσης τιμής & 0.5889702706987753 & 99.21519127634552 \\
  \hline
  Φίλτρο διάμεσης τιμής & 0.42011072522698933 & 78.37677099175538 \\
  \hline
  \end{tabular}
  \caption{Τα αποτελέσματα των συγκρίσεων μετάξυ της εικόνας με τον θόρυβο και της εξομάλυνσης του Poisson}
  \label{tab:results_poisson_noise}
\end{table}

\subsection{Συμπεράσματα}

Ο αλγόριθμος Salt and Pepper φαίνεται να αλλάζει την εικόνα λιγότερο από την Poisson. Παρ'όλ'αυτά, στον Poisson φαίνονται τα χαρακτηριστικά του προσώπου καλύτερα από ότι στο Salt and Pepper.
Στα φίλτρα, το φίλτρο διάμεσης τιμής είναι το καλύτερο φίλτρο, επαναφέρει την εικόνα αρκετά πιο κοντά από ότι στις υπόλοιπες και αυτό φαίνεται στα αποτελέσματα του SSIM και του Mean Square Error.
